\documentclass[a4paper,11pt]{article}
\usepackage[T1]{fontenc}
\usepackage[utf8]{inputenc}
\usepackage{lmodern}

\title{doping discussion}
\author{Ben Rosemeyer}

\begin{document}

\maketitle


\section*{}
We would like to compare this enhancement in electron spin susceptibility with the phase diagram of Cd (hole) doped CeCo(In$_{5-x}$Cd$_x$) with no applied field. Experiments have shown that T$_c$ is suppressed and an AFM phase emerges as doping is increased, and the application of pressure can essentially reverse the doping effects. One interpretation of these results is that pressure and doping both change the chemical pressure, and induce a shift in the fermi level(s). 

If the nature of this AFM state is dominated by electrons which require a $0.1\%$ increase in $\chi_0$ to become magnetically ordered (as discussed here), then the shift in Fermi level would lead to an increased DOS at the Fermi level and corresponding increase of the Pauli susceptibility ($2mu_B^2 N_F$). This picture rules out the perfect 2D case which has a constant density of states. If one considers a 3D free electron gas, ($N(\epsilon) \propto\sqrt{\epsilon}$), then a $0.2\%$ \textit{increase} of the Fermi suface would be neaded to induce magnetic order. 

We can compare this to de Haas-van Alphen experiments where they see a $~2\%$ \textit{decrease} in the Fermi surface volume with $5\%$ nominal doping which rules out the 3D free electron case. Pressure studies of $x=10\%$ crystals indicate that pressures $>2GPa$ are enough to re-establish the SC state as the ground state, possibly restoring the $2\%$ decrease by delocalizing the Ce f-electrons.

Understanding the doping and pressure part of the phase diagram for CeCoIn$_5$ with this model will require an electronic dispersion whose DOS increases for a small decrease in the fermi level. One must also establish exactly how doping and pressure effect the relative localization of the Ce f-electrons and the fermi level.
\end{document}
