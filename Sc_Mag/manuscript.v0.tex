\documentclass[aps,prl,twocolumn,showpacs,amsmath,amssymb]{revtex4-1}
%\documentclass[aps,prl,twocolumn,showpacs,preprintnumbers,amsmath,amssymb,citeautoscript]{revtex4-1}
%\documentclass[prl,showpacs,amssymb,amsmath,twocolumn]{revtex4-1}
%%%%%%%%%%%%
\usepackage{bookmath} % definitions and shortcuts
%%%%%%%%%%%%
\usepackage{graphicx}
\usepackage{amsmath}
\usepackage{color}
\newcommand{\blue}{\textcolor{blue}}
\newcommand{\red}{\textcolor{red}}
\newcommand{\cecoin}{CeCoIn$_5$} 
\def\opp#1{{\overline{ #1}}}


\bibliographystyle{apsrev4-1}
%~~~~~~~~~~~~~~~~~~~~~~~~~~~~~~~~~~~~~~~~~~~~~~~~~~~~~~~~~~~~~~~~~~~~~~~~~~~~~~~%
\begin{document}
%~~~~~~~~~~~~~~~~~~~~~~~~~~~~~~~~~~~~~~~~~~~~~~~~~~~~~~~~~~~~~~~~~~~~~~~~~~~~~~~%
\title{Electronic Spin Susceptibility Enhancement in Pauli Limited Unconventional Superconductors}

\author{Ben Rosemeyer}
\author{Anton~B.~Vorontsov}

\affiliation{Department of Physics, Montana State University, Montana 59717, USA}

\date{\today}

\begin{abstract}
%
%
\end{abstract} 

%\pacs{74.20.Rp,74.25.Dw, \dots} % \red{MORE?,DIFFERENT?} }

%74.20.Rp 	Pairing symmetries (other than s-wave) 
%74.25.Dw 	Superconductivity phase diagrams 


\maketitle

\section{Hamiltonian}


The matrix representation($\mathcal{H} = \bar{\Psi}\mathcal{\hat{H}} \Psi$) of the Hamiltonian of interest is:


\begin{widetext}
\bea
\label{eq:modelH} 
\mathcal{\hat H} = \left(
	\begin{array}{cccc} \xi_{\vk-} \vsigma^0-e_z \vsigma^z & \Delta_{\vk-}(i\vsigma^y) & m^*_\vq(\vsigma^m)^\dagger & 0 \\ 
				       -\Delta^*_{\vk-}(i\vsigma^y) & -(\xi_{-\vk-} \vsigma^0-e_z \vsigma^z) & 0 & -m^*_\vq\big((\vsigma^m)^T\big)^\dagger \\ 
				       m_\vq \vsigma^m & 0 & \xi_{\vk+} \vsigma^0-e_z \vsigma^z & \Delta_{\vk+}(i\vsigma^y) \\ 
				       0 & -m_\vq(\vsigma^m)^T & -\Delta^*_{\vk+}(i\vsigma^y) & -(\xi_{-\vk+} \vsigma^0-e_z \vsigma^z)
	\end{array}
\right)
\eea
\end{widetext}
\begin{widetext}
\bea
\label{eq:modelPsi} 
\Psi^\dagger= \left[
	\begin{array}{cccccccc} c^\dagger_{\vk-\uparrow}, & c^\dagger_{\vk-\downarrow}, & c_{-\vk-\uparrow}, & c_{-\vk-\downarrow}, & c^\dagger_{\vk+\uparrow}, & c^\dagger_{\vk+\downarrow}, & c_{-\vk+\uparrow}, & c_{-\vk+\downarrow} \\ 
	\end{array}
\right]
\eea
\end{widetext}
Where $\vk\pm = \vk \pm \vq/2$, and $\vsigma^i$ are the pauli spin matricies with $\vsigma^0 = \mathcal{I}$ and $\vsigma^m = \vsigma \cdot \hat m$. The Zeeman energy splitting for electons is $e_z = \mu_e H_0$, with a resulting spectral gap of $2e_z$. Now we fix the magnetic response to be transverse to the large applied field $H_0$, so that $\vsigma^m = \vsigma^x$, which is degenerate with $\vsigma^m = \vsigma^y$. The eigenvalues of this matrix can be found by finding the roots of the characteristic polynomial:

\bea
\label{eq:ploly} 
\lambda^4 + p \lambda^2 + q \lambda + r = 0
\eea

where we have used the following definitions

\bea
\label{eq:modelDefs} 
p = -(\Sigma_{\vk-}^2 + \Sigma_{\vk+}^2) \\
q = \pm2e_z(E^2_{\vk-} - E^2_{\vk+}) \\
r = E^2_{\vk-}E^2_{\vk+} - 2m^2 E_q^2 + m^4 \\
\Sigma^2_\vk = \xi^2 + \Delta_\vk^2 + e_z^2 + m_q^2 \\
E^2_\vk = \xi^2_\vk + \Delta^2_\vk - e_z^2 \\
E^2_\vq = \xi_{\vk-}\xi_{\vk+} +\Delta_{\vk-}\Delta_{\vk+} - e_z^2
\eea

The corresponding eigenvectors are:
%%%%%%%%%%%%%%%%%%%%%%%%%%%%%%%%%%%%%%%%%%%%%%%%
%%%%%%%%%%%%%%%%%%%%%%%%%%%%%%%%%%%%%%%%%%%%%%%%
\begin{widetext}
\bea
\Psi^+= \left[
	\begin{array}{c} 
		\Delta_{\vk+} + \frac{ (-(\Delta_{\vk-}\Delta_{\vk+} - m_\vq^2)  - (-e_z-\xi_{\vk-}-\lambda)(e_z-\xi_{\vk+}-\lambda))(e_z +\xi_{\vk+} -\lambda)}{\Delta_{\vk+}(-e_z-\xi_{\vk-}-\lambda) + \Delta_{\vk-}(e_z+\xi_{\vk+}-\lambda)} \\
		0 \\
		0 \\
		- \frac{\Delta_{\vk-}(-e_z^2+\Delta^2_{\vk+} +\xi^2_{\vk+}+2e_z\lambda-\lambda^2) - \Delta_{\vk+}m^2_\vq }     {\Delta_{\vk+}(-e_z-\xi_{\vk-}-\lambda) + \Delta_{\vk-}(e_z+\xi_{\vk+}-\lambda)} \\
		0 \\
		\frac{ m_\vq\big[(\Delta_{\vk-}\Delta_{\vk+} - m_\vq^2)  + (-e_z-\xi_{\vk-}-\lambda)(e_z-\xi_{\vk+}-\lambda)\big]}     {\Delta_{\vk+}(-e_z-\xi_{\vk-}-\lambda) + \Delta_{\vk-}(e_z+\xi_{\vk+}-\lambda)} \\
		m_\vq \\
		0
	\end{array}
\right]
\eea
%%%%%%%%%%%%%%%%%%%%%%%%%%%%%%%%%%%%%%%%%%%%%%%%
\bea
\Psi^-= \left[
	\begin{array}{c} 
		0 \\
		-\Delta_{\vk+} - \frac{ (-(\Delta_{\vk-}\Delta_{\vk+} - m_\vq^2)  - (e_z-\xi_{\vk-}-\lambda)(-e_z-\xi_{\vk+}-\lambda))(-e_z +\xi_{\vk+} -\lambda)}{\Delta_{\vk+}(e_z-\xi_{\vk-}-\lambda) + \Delta_{\vk-}(-e_z+\xi_{\vk+}-\lambda)} \\
		 -\frac{\Delta_{\vk-}(-e_z^2+\Delta^2_{\vk+} +\xi^2_{\vk+}-2e_z\lambda-\lambda^2) - \Delta_{\vk+}m^2_\vq  }{\Delta_{\vk+}(e_z-\xi_{\vk-}-\lambda) + \Delta_{\vk-}(-e_z+\xi_{\vk+}-\lambda)} \\
		0 \\
		\frac{ -m_\vq\big[(\Delta_{\vk-}\Delta_{\vk+} - m_\vq^2)  + (e_z-\xi_{\vk-}-\lambda)(-e_z-\xi_{\vk+}-\lambda)\big]}{\Delta_{\vk+}(e_z-\xi_{\vk-}-\lambda) + \Delta_{\vk-}(-e_z+\xi_{\vk+}-\lambda)} \\
		0 \\
		0 \\
		m_\vq 
	\end{array}
\right]
\eea
\end{widetext}
%%%%%%%%%%%%%%%%%%%%%%%%%%%%%%%%%%%%%%%%%%%%%%%%
%%%%%%%%%%%%%%%%%%%%%%%%%%%%%%%%%%%%%%%%%%%%%%%%
For $q = 2e_z(E^2_{\vk-}-E^2_{\vk+})$ and $q = -2e_z(E^2_{\vk-}-E^2_{\vk+})$ respectively, where $\lambda_i$ is a root of the corresponding polynomial from (\ref{eq:poly}).
%%%%%%%%%%%%%%%%%%%%%%%%%%%%%%%%%%%%%%%%%%%%%%%%
%%%%%%%%%%%%%%%%%%%%%%%%%%%%%%%%%%%%%%%%%%%%%%%%
\begin{widetext}
\bea
\psi_i^+= \left[
	\begin{array}{c} 
		\Delta_{\vk+} - \frac{ (E^2_\vq- m_\vq^2 +e_z(\xi_{\vk+}-\xi_{\vk-})  +\lambda_i(\xi_{\vk+}+\xi_{\vk-})+\lambda_i^2)(\xi_{\vk+}+e_z -\lambda_i)}{\Delta_{\vk-}(\xi_{\vk+}+e_z-\lambda_i)-\Delta_{\vk+}(\xi_{\vk-}+e_z+\lambda_i) } \\
		- \frac{\Delta_{\vk-}(E^2_{\vk+}+2e_z\lambda_i-\lambda_i^2) - \Delta_{\vk+}m^2_\vq }     {\Delta_{\vk-}(\xi_{\vk+}+e_z-\lambda_i)-\Delta_{\vk+}(\xi_{\vk-}+e_z+\lambda_i) } \\
		\frac{ m_\vq\big[E^2_\vq- m_\vq^2  + e_z(\xi_{\vk+}-\xi_{\vk-}) +\lambda_i(\xi_{\vk+}+\xi_{\vk-})+\lambda_i^2\big]}     {\Delta_{\vk-}(\xi_{\vk+}+e_z-\lambda_i)-\Delta_{\vk+}(\xi_{\vk-}+e_z+\lambda_i) } \\
		m_\vq 
	\end{array}
\right]
\eea
%%%%%%%%%%%%%%%%%%%%%%%%%%%%%%%%%%%%%%%%%%%%%%%%
\bea
\psi^-= \left[
	\begin{array}{c} 
		-\Delta_{\vk+} + \frac{ (E_\vq- m_\vq^2  - e_z(\xi_{\vk+}-\xi_{\vk-})+\lambda_i(\xi_{\vk+}+\xi_{\vk-})+\lambda_i^2))(\xi_{\vk+}-e_z -\lambda_i)}{\Delta_{\vk-}(\xi_{\vk+}-e_z-\lambda_i)-\Delta_{\vk+}(\xi_{\vk-}-e_z+\lambda_i) } \\
		 -\frac{\Delta_{\vk-}(E^2_{\vk+}-2e_z\lambda_i-\lambda_i^2) - \Delta_{\vk+}m^2_\vq  }{\Delta_{\vk-}(\xi_{\vk+}-e_z-\lambda_i)-\Delta_{\vk+}(\xi_{\vk-}-e_z+\lambda_i) } \\
		-\frac{ m_\vq\big[E^2_\vq- m_\vq^2  - e_z(\xi_{\vk+}-\xi_{\vk-}) +\lambda_i(\xi_{\vk+}+\xi_{\vk-})+\lambda_i^2\big]}{\Delta_{\vk-}(\xi_{\vk+}-e_z-\lambda_i)-\Delta_{\vk+}(\xi_{\vk-}-e_z+\lambda_i) } \\
		m_\vq 
	\end{array}
\right]
\eea
\end{widetext}
%%%%%%%%%%%%%%%%%%%%%%%%%%%%%%%%%%%%%%%%%%%%%%%%
%%%%%%%%%%%%%%%%%%%%%%%%%%%%%%%%%%%%%%%%%%%%%%%%
These vectors can be written as two different sets of four vectors corresponding to the basis states

\bea
 \left[
	\begin{array}{cccc} c^\dagger_{\vk-\uparrow},  & c_{-\vk-\downarrow},  & c^\dagger_{\vk+\downarrow}, & c_{-\vk+\uparrow} \\ 
	\end{array}
\right] 
\eea
and
\bea
\left[
	\begin{array}{cccccccc}  c^\dagger_{\vk-\downarrow}, & c_{-\vk-\uparrow},  & c^\dagger_{\vk+\uparrow},  & c_{-\vk+\downarrow} \\ 
	\end{array}
\right]
\eea

Now we can write the matrix which transforms to these new basis states $\Gamma = \{\Psi_i^+, \Psi_i^-\}$,  $\mathcal{D}\Psi = \Gamma$ with operators $\gamma^{\pm}_{\vk, i}$:


\bea
\mathcal{D}^+= \left[
	\begin{array}{c} 
		(\psi_1^+)^\dagger \\
		(\psi_2^+)^\dagger \\
		(\psi_3^+)^\dagger \\
		(\psi_4^+)^\dagger 
	\end{array}
\right] \quad,\quad\quad\quad
\mathcal{D}^-= \left[
	\begin{array}{c} 
		(\psi_1^-)^\dagger \\
		(\psi_2^-)^\dagger \\
		(\psi_3^-)^\dagger \\
		(\psi_4^-)^\dagger 
	\end{array}
\right]
\eea

\bea
 \left[
	\begin{array}{cccc} \gamma^\dagger_{\vk-1},  & \gamma_{-\vk-2},  & \gamma^\dagger_{\vk+3}, & \gamma_{-\vk+4} \\ 
	\end{array}
\right] 
\eea
and
\bea
\left[
	\begin{array}{cccc}  \gamma^\dagger_{\vk-1}, & \gamma_{-\vk-2},  & \gamma^\dagger_{\vk+3},  & \gamma_{-\vk+4} \\ 
	\end{array}
\right]
\eea


By computing the inverse of $\mathcal{D}$ we can find the old operators in terms of the new to use in the self consistent calculation for $\Delta$ and $m_\vq$.



We also wish to calculate the free energy (F) in this new mixed state $F= -T ln(Z)$ where $Z$ is the partition function for the diagnol Hamiltonian ($\mathcal{H} = \sum\limits_{i} \lambda_i \gamma^\dagger_i\gamma_i$):
\bea
Z &    = Tr\bigg( e^{-\beta(\mathcal{H} - \mu N)} \bigg) \\
	 & \Pi_{\vk,i} \bigg(  1 +  e^{-\beta\lambda_{\vk, i}} \bigg) \\
F &= -T \sum\limits_{\vk,i} ln\big(1 + e^{-\beta\lambda_{\vk,i}}\big)
\eea



\section{Characteristic Polynomial}
 - Add and subtract $\lambda^2 u + u^2/4$
 \bea
 \lambda^4 + p\lambda^2 +q\lambda+r \\
 = (\lambda^2 + \frac{u}{2})^2 - ( (u-p)\lambda^2 -q\lambda +(\frac{u^2}{4} - r)) \\
 = (\lambda^2 + \frac{u}{2})^2 - (u-p)\bigg{(}\lambda^2 -\frac{q}{u-p}\lambda +\frac{\frac{u^2}{4} - r}{u-p}\bigg{)} \\
  = (\lambda^2 + \frac{u}{2})^2 - (u-p)\bigg{(}\lambda^2 -\frac{q}{u-p}\lambda +\frac{\frac{u^2}{4} - r}{u-p}\bigg{)} 
 \eea
 
 \bea
 P = (\lambda^2 + \frac{u}{2}) \\
 Q = \sqrt{u-p} \bigg( \lambda -\frac{q}{2(u-p)}\bigg)
  \eea
  Which requires
  \bea
  \frac{q}{2(u-p)} = \sqrt{ \frac{u^2/4-r}{u-p}} \\
  q^2/4 = (u-p)(u^2/4-r)  \\
  q^2/4 = u^3/4 -pu^2/4 -ru +rp \\
  0 = u^3-pu^2 -4ru +(4rp-q^2)
  \eea
 Now we get $\lambda^4 + p \lambda^2 + q \lambda + r = (P+Q)(P-Q)  = P^2 - Q^2$ for a certain $u$ which satisfies the resolvent cubic (above, only need one root)
 We can find the solutions to the cubic exactly by the following method: \\
 1)	write above as $u^3 + c_2 u^2 + c_1 u + c_0$ \\
 	$c2 = -p$ \\
	$c1 = -4r$\\
	$c0 = 4rp - q^2$\\
 2)	define $u = z-c_2/3$ to get $z^2 + a_1z = a_0$ \\
	$ a_1 = c_1 -  c_2^2/3$ \\
	$a_0  = c_1c_2/3 - c_0 - 2c_2^3/27$ \\
 3)	define $ z = w - a_1/(3w)$ to get $(w^3)^2 - a_0w^3 - a_1^3/27 = 0$ \\
 The solutions to w are:
 \bea
 w = \sqrt[3]{\frac{a_0}{2} \pm \sqrt{\bigg(\frac{a_0}{2}\bigg)^2  + \frac{a_1^3}{27} }}
 \eea
 The corresponding solutions to u are:
 \bea
 u = w - a_1/(3w) - c_2/3
 \eea
 
 
% The solutions come from the cubic formula:
% \bea
% u_1 = -a_2/3 + (S + T) \\
% u_{2,3} =-a_2/3-\frac{1}{2}(S+T)\pm\frac{i\sqrt{3}}{2}(S-T)
% \eea
% with the definitions:
 
% \bea
% S = \sqrt[3]{R+\sqrt{D}} \\
% T = \sqrt[3]{R-\sqrt{D}} \\
% D = Q^3 +R^2 \\
% R = \frac{9a_2a_1-27a_0-2a_2}{54} \\
% Q = \frac{3a_1-a_2^2}{9}
% \eea
%\bea
%a_2 = -p \\
%a_1 = -4r \\
%a_0 = 4pr-q^2
%\eea

Plug in u and solve for roots of $P\pm Q$:
%\bea
%\lambda^2 \pm \sqrt{u-p}\lambda + \frac{1}{2}\bigg(u \mp\frac{q}{\sqrt{u-p}}\bigg) \\
%\lambda = \pm\frac{\sqrt{u-p}}{2} \{\pm\} \sqrt{ \frac{u-p}{4} -\frac{1}{2}\bigg(u \pm\frac{q}{\sqrt{u-p}}\bigg)}
%\eea
\bea
\lambda^2 \pm \sqrt{u-p}\lambda + \frac{1}{2}\bigg(u \mp{\sqrt{u^2-4r}}\bigg) \\
\lambda = \pm\frac{\sqrt{u-p}}{2} \{\pm\} \sqrt{ \frac{|u-p|}{4} -\frac{1}{2}\bigg(u \pm\sqrt{u^2-4r}\bigg)}
\eea



 %~~~~~~~~~~~~~~~~~~~~~~~~~~~~~~~~~~~~~~~~~~~~~~~~~~~~~~~~~~~~~~~~~~~~~~~~~~~~~~~
  %~~~~~~~~~~~~~~~~~~~~~~~~~~~~~~~~~~~~~~~~~~~~~~~~~~~~~~~~~~~~~~~~~~~~~~~~~~~~~~~
   %~~~~~~~~~~~~~~~~~~~~~~~~~~~~~~~~~~~~~~~~~~~~~~~~~~~~~~~~~~~~~~~~~~~~~~~~~~~~~~~
    %~~~~~~~~~~~~~~~~~~~~~~~~~~~~~~~~~~~~~~~~~~~~~~~~~~~~~~~~~~~~~~~~~~~~~~~~~~~~~~~
     %~~~~~~~~~~~~~~~~~~~~~~~~~~~~~~~~~~~~~~~~~~~~~~~~~~~~~~~~~~~~~~~~~~~~~~~~~~~~~~~
      %~~~~~~~~~~~~~~~~~~~~~~~~~~~~~~~~~~~~~~~~~~~~~~~~~~~~~~~~~~~~~~~~~~~~~~~~~~~~~~~

\bibliography{mybib}

%~~~~~~~~~~~~~~~~~~~~~~~~~~~~~~~~~~~~~~~~~~~~~~~~~~~~~~~~~~~~~~~~~~~~~~~~~~~~~~~%
\end{document}
%~~~~~~~~~~~~~~~~~~~~~~~~~~~~~~~~~~~~~~~~~~~~~~~~~~~~~~~~~~~~~~~~~~~~~~~~~~~~~~~%
