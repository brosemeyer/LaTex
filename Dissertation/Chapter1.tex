\chapter{\label{chap:1}Introduction}

The phenomenon of superconductivity (SC) occurs when electrons Bose condense below some critical temperature $T_c$, forming ``Cooper pairs'' of time reversed electrons. How these Cooper pairs behave respond to magnetic interactions has been an important topic for condensed matter physics. 

Quantum mechanics requires the spin wave function of the pair to be either a singlet ($S=0$) or triplet ($S=1$). The singlet state is subject to Pauli pair breaking, when one of the electron spins flips to align with the field, destroying the singlet. Pauli limiting reduces the upper critical field $H_{c2}$ compared to orbital limiting, where the opposite orbital motion of the time reversed pairs breaks the Cooper pair. Triplet superconductors are often orbitally limited.

In spite of this limiting behavior, new materials and experimental results indicate that in unconventional or non-uniform superconductors there may be coexistent magnetic order and superconductivity.

\section{\label{sec:1.1}Historical Background}

The antagonistic relation between SC and magnetism had been known to experiment even before the generally accepted microscopic theory by Bardeen, Cooper and Schriefer (BCS theory) was born in 1956-57. The first evidence came about 20 years after Kamerlingh Onnes discovered superconductivity in mercury (1911). Walther Meissner and Robert Ochsenfeld found that a magnetic field applied to a superconductor is expelled from the bulk, a phenomenon now known as the Meissner effect. Shortly thereafter, Fritz and Heinz London used electromagnetic theory to establish the London equations which stated that the magnetic field is cancelled by supercurrents circulating on the surface. Ginzburg also concluded that the magnetism must be zero in the bulk of type I superconductors. And so began the trend of magnetism and superconductivity being mutually exclusive.

In 1959, using the BCS theory, Anderson and Suhl showed that the magnetic susceptibility at zero wave-vector ($\chi(\vq)$, $\vq\rightarrow 0$) is suppressed by Cooper pairs and the resulting gapped density of states. This further ruled out the possibility of an RKKY interaction leading to magnetism. The RKKY theory attempts to describe the interactions between the localized moments of ions,  mediated by the electron gas. The ferromagnetic transition temperature in this theory is linearly proportional to $\chi(0)$.

Many theorists considered a magnetic exchange between electrons or with the local ion moments to find that the exchange field of a magnetic state has a paramagnetic effect, which tends to align the spins of electrons and break the singlet pair, destroying superconductivity. Yet others thought that the ion moments might be able to become magnetically ordered via a dipole-dipole interaction, but found that strong Meissner screening currents around the ion moments caused the long-range interaction to be small.

The above evidences made a strong case for the exclusive nature of superconductivity and ferromagnetism, but not anit-ferromagnetism (AFM). For magnetic ordering wave-vectors which are large enough compared to the inverse coherence length of the cooper pairs ($q>>\xi_c$), the average magnetism will be nearly zero on the coherence length scale. On this basis, many scientists concluded that a coexistent AFM-SC state was possible, a fact that has been experimentally confirmed, and shown to arise from exchange or electromagnetic mechanisms.

From then on, the AFM-SC state was a playground for theorists who were able to craft many scenarios where the two phenomenon could coexist. Further advances came as more materials where discovered as type II. Unlike Type I, Type II superconductors allow magnetic fields to penetrate the bulk as an array of individual field lines, each with its own circulating supercurrents.


\section{\label{sec:1.2}Unconventional and non-uniform Superconductivity}

The discovery of unconventional superconductivity in the 1980's and it's mysteries further livened the field of condensed matter physics. These materials include heavy fermion materials, high $T_c$ cuprates, organic superconductors and most recently iron based superconductors. These materials are of high importance to our understanding of superconductivity and magnetism. The normal state(s) usually have electronic structure near the Fermi level which has reduced dimensionality, and most theories assume quasi-two dimensional behavior of the Cooper pairs. Additionally, the electrons are often non-Fermi like above $T_c$, indicating a possible phase transition which is masked by superconductivity. Many believe this hidden phase to be magnetic.

They are further characterized by an anisotropic order parameter which has nodes where $\Delta(\vk)=0$. The nodal nature of $\Delta$ was first confirmed in experiment when Ott et al. noticed the $T^3$ dependence of specific heat in heavy fermion $UBe_{13}$ which is consistent with point nodes. Scientists soon began to draw comparisons between unconventional superconductors and superfluid $^3He$ which also has nodes in its' order parameter, and is known to have strong spin fluctuations. However, there efforts only went so far and the non-Fermi like normal state and mechanism for superconductivity continues to by a mystery. 

The cuprates and iron based superconductors fall into the high $T_c$ catagory of superconductivity with critical temperatures as high as 150K. The commonality between these two is the square planar crystal structure ($CO$ planes or $Fe$) where SC electrons are believed to exist, and an antiferromagnetic ground state from which superconductivity manifests. Researchers quickly began formulating theories which include both magnetism and superconductivity in the crystalline  basis. But like the heavy fermions, many questions are unanswered.

One open question is the ``pseuo-gap'' state that appears near the superconducting transition. In this state, the density of states is nearly gapped (instead of fully gapped as in SC). Much effort has be given to understanding this as well, and many theories include magnetic interactions with superconductivity.

Unconventional superconductivity has also revived the field of non-uniform SC. The theory has been established in the BCS framework independently by the groups of Fulde-Ferrel and Larkin-Ovchinikov. It predicts that singlet superconductors which are strongly Zeeman coupled can avoid the Pauli limit by pairing non-time reversed electrons ($\vk$ with $-\vk+\vq$ instead of $\vk$ with $-\vk$). The pairs then have finite momentum $\vq$ and cause the system to find a new ground state order parameter that oscillates in real space. Generally this state is referred to as FFLO, and the resulting structure has nodes of the order parameter which can harbor spatially confined Andreev bound states with energy below the bulk gap. These bound states have a myriad of special properties including being their own anti-particle and could be very important in the formation of magnetic order. 

\cecoin is a heavy fermion material and has a coexistent AFM-SC state at high fields and low temperature that may be FFLO. This state has sparked much interest, and the question remains if the order is due to it's AFM normal state tendencies, which include magnetically active f electrons from $Ce$, or formation of a more exotic FFLO-AFM state.


Besides \cecoin , another leading candidate for FFLO is the organic superconductor \kbtf. Recent experiments have detected an increase in the spin-lattice relaxation rate \relt at high fields near $H_{c2}$ and there is a possibility that Andreev bound states present in the system are the cause.

\section{\label{sec:1.3}Thesis Organization}

In Chapter 2 we will develop the theoretical background and mathematics necessary for a mean field treatment of the SC state for uniform and non-uniform cases. The linear response theory for magnetism will be established and results from other theories which include magnetic interactions will also be discussed here.

Chapter 3 concerns the calculation of electron spin susceptibility for unconventional superconductors at low temperature and high applied fields near $H_{c2}$ for a Pauli limited SC. Using symmetry arguments we identify antiferromagnetic ordering wave-vectors that enhance the susceptibility beyond the normal state Pauli limit and compare with numeric results.

In Chapter 4 we consider the non-uniform state for both conventional S-wave and unconventional D-wave SC. We calculate spin susceptibility, but now as a function of center of mass coordinate also. We find a dramatic increase in $\chi$ near a domain wall where $\Delta$ is suppressed to zero and has opposite sign on either side of the wall. This behavior is studied using the Andreev approximation to gain insight into the symmetry conditions and amplitude. 

Chapter 4 also includes the calculation of spin-lattice relaxation rate near a domain wall. We again find a large increase and attribute it to the presence of bound states. It is quickly suppressed in a Zeeman field as the bound states become shifted away from zero energy and out of the thermally allowed excitation energy range of the distribution function. There is a re-emergent region at higher fields with relaxation rate above the normal limit, due to the competition between thermal smearing and Zeeman splitting of the distribution function.

In Chapter 5 we conclude with remarks about experimental observations and future work.