\documentclass[a4paper,11pt]{article}
\usepackage[T1]{fontenc}
\usepackage[utf8]{inputenc}
\usepackage{lmodern}
\usepackage{bookmath}
\usepackage{amsmath}
\usepackage{graphicx}
\title{HW 12: Physics 545}
\author{Ben Rosemeyer}

\begin{document}

\maketitle

\section*{1a}
FIGURE ATTACHED

The density of states $D(\epsilon)$ for this 2D dispersion of graphene is:
\bea
D(\epsilon) &=& \sum\limits_{\vk s}\delta(\epsilon - \epsilon_{\vk s}) \\
 &=& 2\pi V/(2\pi\hbar)^2\int dk \quad k(\delta(\epsilon - v_f k) +\delta(\epsilon + v_f k) ) \\
  &=& D_0 |\epsilon|
\eea
with $D_0 = \frac{V}{2\pi\hbar^2v_f^2}$, and we note that it is independent of spin $s$

Also, it's kind of "gapped" in that there are no states at zero energy.

\section*{1b}
At zero temperature, with $\mu(0)=0$, the contribution to number of electrons is only from $s=-1$ because the Fermi distribution function $f_{\vk 1}=0$ and $f_{\vk -1}=1$
\bea
N_0 &= & D_0\int\limits_{0}^\infty d\epsilon \quad \epsilon
\eea
The above is divergent, but we can still compare with the electron number for finite temperature:

\bea
N(T) &=&\sum_{\vk s} (e^{\beta (\epsilon_{\vk s}-\mu)} + 1)^{-1} \\
 &=& D_0\int\limits_{-\infty}^\infty d\epsilon \quad |\epsilon| (e^{\beta (\epsilon-\mu)} + 1)^{-1} \\
  &=& D_0\int\limits_0^\infty d\epsilon \quad \epsilon \bigg[1 + (e^{\beta (\epsilon-\mu)} + 1)^{-1} - (e^{\beta (\epsilon+\mu)} + 1)^{-1}\bigg] \\
  &=& N_0 + D_0\int\limits_0^\infty d\epsilon \quad \epsilon \bigg[(e^{\beta (\epsilon-\mu)} + 1)^{-1} - (e^{\beta (\epsilon+\mu)} + 1)^{-1}\bigg] \\
  &= & N_0 + D_0 \bigg[\int\limits_{-\mu}^\mu dx \quad \frac{x+\mu}{e^{\beta x} + 1} + 2\mu \int\limits_{\mu}^\infty dx \quad \frac{1}{e^{\beta x} + 1}\bigg]
\eea

For conservation of particles we require $N(T) = N_0$. Therefore, the term in brackets (8) must be zero, so $\mu = 0$

\section*{1c}
To calculate the specific heat we use the form $C = \frac{dU}{dT}$, U is the internal energy.

\bea
 U &=& D_0\int\limits_{-\infty}^\infty d\epsilon \quad|\epsilon|\epsilon f(\epsilon) \\
&=& D_0\int\limits_{0}^\infty d\epsilon \quad \epsilon^2( f(\epsilon) -f(-\epsilon)) \\ 
&=& D_0\int\limits_0^\infty d\epsilon \quad\epsilon^2\,\,(2f(\epsilon)-1) \\
&=& 2 T^3 D_0\int\limits_0^\infty dx \quad \frac{x^2}{e^x+1} - D_0\int\limits_0^\infty d\epsilon \quad\epsilon^2
\eea
The second term is the ground state energy $E_0$ and does not contribute to the specific heat because there is no temperature dependence there.

The integral in the first term can be found in tables and the result is
\be
\delta U = 3 T^3 D_0\zeta(3) + E_0
\ee

\be
C = 9 T^2 D_0\zeta(3)
\ee

\section*{2}
The equation for minimum of energy is:
\be
\frac{\partial F}{\partial \veta} = 0=\sum\limits_{i=1,2} \alpha \eta_i + \beta_1 (\veta\cdot\veta^*)\eta_i + \beta_2(\veta\cdot\veta)\eta^*_i + \beta_3 |\eta_i|^2\eta_i
\ee

for the 4 orientation cases to consider we can write the above as a function of magnitude $\eta_{01}$, $\eta_{10}$, $\eta_{11}$, $\eta_{1i}$, noting that $\eta_{01}$ and  $\eta_{10}$ are the same. 

\bea
0 =  \alpha \eta_{10} + \beta_1 |\eta_{10}|^2\eta_{10} + \beta_2 \eta_{10}^2 \eta^*_{10} + \beta_3 |\eta_{10}|^2\eta_{10} \\
\Rightarrow |\eta_{10}|^2 = \frac{-\alpha}{\beta_1 + \beta_2 + \beta_3}
\eea

\bea
0 =  \alpha \eta_{11} + 2\beta_1 |\eta_{11}|^2\eta_{11} + 2\beta_2 \eta_{11}^2 \eta^*_{11} + \beta_3 |\eta_{10}|^2\eta_{11} \\
\Rightarrow |\eta_{11}|^2 = \frac{-\alpha}{2\beta_1 + 2\beta_2 + \beta_3}
\eea

\bea
0 =  \alpha \eta_{1i} + 2\beta_1 |\eta_{1i}|^2\eta_{1i} + \beta_3 |\eta_{1i}|^2\eta_{1i} \\
\Rightarrow |\eta_{1i}|^2 = \frac{-\alpha}{2\beta_1  + \beta_3}
\eea

The free energy of these states is:

\bea
F[\eta_{10}] = \frac{-\alpha^2/(2\beta_1)}{1 + \beta_2/\beta_1 + \beta_3/\beta_1}
\eea

\bea
F[\eta_{11}] = \frac{-\alpha^2/(\beta_1)}{2 + 2\beta_2/\beta_1 + \beta_3/\beta_1}
\eea

\bea
F[\eta_{1i}] = \frac{-\alpha^2/(\beta_1)}{2  + \beta_3/\beta_1}
\eea

For each possible value of $x=\beta_2/\beta_1$ and $y=\beta_3/\beta1$ the system will choose the state with lowest free energy. The conditions on this choice are
{\bf choose 10 over 11}: $y<0$ \\
{\bf choose 10 over 1i}: $y<-2x$ \\
{\bf choose 11 over 1i}: $x<0$ \\
The regions are mapped out in the figure

\section*{3a}
The diagonalized Hamiltonian is:
\be
\cH = \sum_{\vk,s}E_{\vk s}b_{\vk s}^\dagger b_{-\vk s}
\ee
and the $b$ operators are defined through the Bogolioubov transformation:
\bea
a_{\vk s} = u_{\vk} b_{\vk s} - (i\sigma_y)_{ss'} v_{\vk} b_{-\vk s'}
\eea
which results in the Bogolioubov-de Gennes equations
\bea
E_{\vk s}\left(
\begin{array}{cc}
 u_{\vk} \\ v_{\vk} 
\end{array}
\right) = \left(
\begin{array}{cc}
\xi_{\vk}  & \Delta^*\\ \Delta & \xi_{\vk \bar{s}}  
\end{array}
\right)
\left(
\begin{array}{cc}
 u_{vk} \\ v_{\vk} 
\end{array}
\right)
\eea

In the above definition for $b$ operators I omitted spin indices on the amplitudes in anticipation of the final result which is
\bea
u_{\vk} = \sqrt{\frac{\Delta}{E}} e^{\theta_E/2} \\
v_{\vk} = \sqrt{\frac{\Delta}{E}} e^{-\theta_E/2}
\eea
The eigenvalues are
\be
E_{\vk s} = \sqrt{\xi_{\vk}^2 + |\Delta|^2} - \mu_B H s
\ee

with $\theta_E = cosh^{-1}(E/\Delta)$.
\section*{3b}

After using the form of $a$ operators in terms of the diagonal $b$'s, the spin magnetisation reduces to
\be
M = \mu_B \sum\limits_{\vk} f_{\vk \uparrow} - f_{\vk \downarrow} 
\ee
where $f_{\vk s}=(e^{\beta E_{\vk s}} +1)^{-1}$ and $\beta = 1/T$. This is good! 

We can preform the sum over states by using the density of states for a superconductor $N = N_0|E|/\sqrt{E^2 - \Delta^2}$ and $N_0$ is the normal DOS at the Fermi level.

\bea
M &=& \mu_B N_0 \int\limits_{\Delta}^\infty dE \frac{E}{\sqrt{E^2-\Delta^2}}( f_{\vk \uparrow} - f_{\vk \downarrow}) 
\eea

The linear response to a field will be $M_{lr} = \chi(T) H$ and we can Taylor expand the Fermi functions in H $f_{\vk s}\approx f_{\vk} + \frac{s\beta}{4cosh^2(\beta E/2)} H $ to get the desired result
\bea
\chi(T) &=& (1/2)\mu_B N_0\beta \int\limits_{\Delta}^\infty dE \frac{E}{\sqrt{E^2-\Delta^2}}\frac{1}{cosh^2(\beta E/2)} \\
	 &=& (1/2)\mu_B N_0 \int\limits_{\beta\Delta}^\infty dx \frac{x}{\sqrt{x^2-(\beta\Delta)^2}}\frac{1}{cosh^2(x/2)} \\
	 &=& (1/2)\mu_B N_0 \int\limits_{\beta\Delta}^\infty dx \sqrt{x^2-(\beta\Delta)^2}\frac{tanh(x/2)}{cosh^2(x/2)}
\eea
and we integrated by parts in the last step
\section*{3c}
The two limits of interest are $\beta \Delta >> 1$, and $\Delta=\Delta_0=constant$ for low temperature and $\beta \Delta << 1$ for near $T_c$ when self consistent solution $\Delta(T)$ is first appearing as a second order phase transition. The limit near $T_c$ would not apply if it was a first order transition.

{\bf $\beta \Delta_0 >> 1$, Low Temperature}\\
The low T limit allows write $tanh(x/2) \approx 1$ and $cosh(x/2)\approx \frac{1}{2} e^{x/2}$
\bea
	\chi(T) &=& 2\mu_B N_0 \int\limits_{\beta\Delta}^\infty dx \sqrt{x^2-(\beta\Delta)^2}e^{-x} \\
	 &=& 2\mu_B N_0 (\beta\Delta)^2 \int\limits_1^\infty dy \sqrt{y^2-1}e^{-\beta\Delta y}
\eea

The integral is equivalent to the $n=1$ modified Bessel function of the second kind.
\be 
K_1(z) = z \int\limits_1^\infty dy \sqrt{y^2-1}e^{-z y}
\ee
\bea
	\chi(T)&=& 2\mu_B N_0 \frac{\Delta_0}{T} K_1(\Delta_0/T)
\eea

{\bf $\beta \Delta << 1$, Near $T_c$}
Here we employ Leibniz rule to Taylor expand the integral 
\bea
\int\limits_{\beta\Delta}^\infty dx \sqrt{x^2-(\beta\Delta)^2}\frac{tanh(x/2)}{cosh^2(x/2)}
&\approx & \int\limits_0^\infty dx \quad x\frac{tanh(x/2)}{cosh^2(x/2)} \\
&-& (\beta\Delta)^2/2\int\limits_0^\infty dx \frac{1}{x}\frac{tanh(x/2)}{cosh^2(x/2)}
\eea

The first integral is easy and evaluates to $2$.

The second integral is not easy, but mathematica can do it and the result is $-28\zeta'(-2)\approx .852557$. At this point we can also use the limiting expression for $\Delta(T)$ near $T_c$  which was written in class.
\be
\Delta(T) = \sqrt{\frac{2}{7|\zeta'(-2)|}}T_c\sqrt{1-T/T_c}
\ee
Now we can finally write the two limits using normal state susceptibility $\chi_N=\mu_B N_0$
\bea
	\frac{\chi^{T_c}(T)}{\chi_N} &=& 1 - 2\frac{1-T/T_c}{(T/T_c)^2}
\eea
\bea
	\frac{\chi^{0}(T)}{\chi_N}&=& 2\frac{\Delta_0}{T} K_1(\Delta_0/T)
\eea

The zero temperature expansion is a very slow growing function of T, and plots show that $\frac{\chi^{0}(T)}{\chi_0} < 0.04$ for $T<0.2\Delta_0$. This is expected from the form of the magnetisation in the beginning of part 3b combined with the gapped density of states.

Near $T_c$ the normal state $\chi_N$ is recovered and the temperature dependence of the decrease in $\chi$ is due to the onset of the order parameter $\Delta(T)$. Looking at plots, one can see that the $\chi$ is quickly reduced to near zero for $T/T_c \approx 0.75$

FIGURES ATTACHED
\end{document}

